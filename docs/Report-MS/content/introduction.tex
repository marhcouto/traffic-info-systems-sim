\section{Introduction}
\label{sec:Introduction}
 
In contemporary urban landscapes, the escalating challenges associated with traffic congestion and inefficient road networks have become critical issues. As cities have continued to expand, so has the demand for effective traffic management strategies. Computer Science has had an exceptionally important role in the development of these strategies, both through the exploration of simulation as a tool for testing these strategies and with the development of algorithms and techniques with the intent of managing the problems presented directly. 

One particular problem in the field of Traffic Management is elegantly depicted by the \textit{Braess Paradox}: counterintuitively, adding a road to a road network could affect negatively its flow i.e. the vehicles' travel time \footnote{\url{https://brilliant.org/wiki/braess-paradox/}}. One possible scenario in which the aforementioned problem appears is the following the on in which the newly added road is shorter in travel time, and therefore more desirable to the drivers. Accordingly, the drivers will prefer this road to the old ones and the vehicles will all choose it, causing congestion in it. In this case, the congestion was generated by a poor strategy for the choice of road by the drivers themselves, as it did not take into consideration the other users of the road or any information other than the road network's static structure i.e. the infrastructure. 

Information percolation has the potential to help define better strategies for the choice of route of a vehicle/driver, as the problem in the scenario described arose due to the driver's lack of knowledge on the road network's state at the time of choice. The project described in this paper aims precisely to explore the influence of information percolation systems as a traffic management strategy by analyzing through simulation the influence of the introduction of such systems for route assignment. A focus will be employed on testing information percolation strategies that benefit from the existence of Connected Autonomous Vehicles, aiming to test the impact that such technology could have by coupling with information dissemination.

The relevance of the work presented resides in the fact that it has the potential to inform on possible strategies based on information percolation for attenuation of congestion in road networks, which improves the quality of life of the communities using the networks. The work should culminate in a simple simulation environment with different modelled strategies, some utilizing the notion of CAVs, some reflecting a scenario already implemented in the real world. Furthermore, it should also originate some conclusions regarding the efficiency of the strategies themselves.

The remaining paper is structured in the following manner, by order:

\begin{enumerate}
    \item Some related work will be presented and analyzed
    \item The methods and materials utilized in the project will be exposed
    \item The results of the project will be displayed and analyzed
    \item The conclusions to be drawn from the results will be given
\end{enumerate}