\section{Conclusion}
\label{sec:Conclusion}

The work presented provides a simulation environment with three different route choice models developed: two descriptive models, using technology already matured at the moment of this experiment, while the third one bases itself on a hypothetical scenario, allowed by the advancements in Autonomous Driving and in the Automotive Industry. All the models developed base the traffic assignment mission on Information Percolation Strategies, which control the amount of information a vehicle has access to and therefore influence the decisions made. The project originates a simple and intuitive environment, from which more work can be built upon. The results indicate the great advantage in the usage of Intelligent Traffic/GPS applications in comparison to regular GPS or maps, as well as the little advantage CAVs could bring in this front in comparison to the aforementioned strategy. This should not, however, discourage further work in the area, as the simulation environment is quite simplistic and makes several assumptions that could hinder the results of the work but were made for the sake of the completion of the work in time.

\subsection{Future Work}

\begin{itemize}
    \item Further validation work
    \begin{itemize}
        \item improve information percolation models fidelity by validating with real-world data e.g. research Google Maps and Intelligent GPS apps further
        \item utilization of road network models that represent real scenarios  
    \end{itemize}
    \item BPR function tuning and adoption of more complex models
\end{itemize}

\section*{Acknowledgment}

This project was developed in the context of the Modelling and Simulation curricular unit, lectured by Professor Rosaldo Rossetti from the Faculty of Engineering of the University of Porto.
