\section{Related Work}
\label{sec:Related-Work}

\subsection{CAVs}

Intelligent transportation systems have seen considerable investment from the scientific community in the last decade. The focus of the work published has been fairly diverse. One of the main thematics reviewed is the inclusion of Connected Autonomous Vehicles in the road networks and the impact these could have, both positive and negative. One important consequence of the inclusion of CAVs in road networks in mass is the environmental changes this could lead to, which is reviewed in \cite{kopelias2020connected}. On the same front, Zhang \cite{zhang2019impact} also analyzes the problems that arise under mixed traffic congestion, focusing on intersections and other complicated scenarios and the impact of CAVs' optimal control on the energy consumption of the vehicles. Other work reviews the impact CAVs could have on road network infrastructures and their geometric elements, outlining the natural reductions of the necessary width of lanes in highways and other equivalent roads. Another principal concern is their impact on road safety. Lanhang Ye \cite{ye2019evaluating} analyzed the impact on road safety of the inclusion of CAVs in road networks at different penetration rates in simulated environments, using the frequency of dangerous situations and collisions as a metric, having concluded that their inclusion was highly beneficial to traffic safety. 

Even so, the main hope for CAVs is for them to greatly influence the ease of traffic flow. Lanhang's work \cite{ye2019evaluating} has also led to the observation, as a byproduct of the analysis, that CAVs were also useful to the ease of the stop-and-go traffic, positively affecting traffic flow in the situations reviewed. Talebpour's \cite{talebpour2016influence} work also supports this possibility, having utilized models that appropriately represent different types of communication technology and concluded that CAVs helped improve string stability and reduce shockwave formation and propagation. This study also indicated the potential of CAVs to improve overall throughput. Ye explored modelling paradigms for CAVs in \cite{ye2018modeling}, which would be important for the development of the models in this project. Ye focuses precisely on the impact of dedicated lanes for CAVs in \cite{ye2018impact}. Some of this study's conclusions once again point towards the benefits of integrating CAVs, indicating, for instance, that it would make sense to set higher speed limits for the CAV lanes, as they are capable of travelling with increased safety. 

\subsection{Vehicle Travel Time Modelling}

To model differences in behaviour on a road network and the choice of route, there is a need to first model the traffic network and obtain travel time as a consequence of the route/road's state and congestion. The \textit{Bureau of Public Roads} (BPR) function is a simple function widely used to determine vehicle travel time as a function of the current volume of the route and its capacity, commonly referred to as a link-cost function. Its name stems from the fact that it is the function used by the \textit{Bureau of Public Roads}, now \textit{Federal Highway Administration}, from the U.S. Department of Transportation. The work of \cite{gore2023modified} presents a modern version of this function, that aims to model the stochastic nature of the problem. However, the original function will be used in this project, as the added complexity it brings did not clearly benefit the study.  

\subsection{Novelties of this work}

Many of the studies carried out in the last decade focus on mixed scenarios, as have the works cited above, where manual vehicles still occupy the road networks, meaning they do not envision a penetration rate of 100\%, such as \cite{ye2018impact}. This work differs from the ones referenced in exactly this aspect, as we explore a scenario with a 100\% penetration rate of CAVs. Moreover, many of the studies made focus on the subtle behavioural differences inside a certain road or route, or in complicated specific scenarios such as junctions and intersections. The simulation developed in this project operates at a higher level of abstraction, focusing on the impact of CAVs on route choice and its consequences on overall network throughput. Moreover, the project focuses on understanding the advantages of these techniques compared to non-CAV base cases, whereas the most common works simply study the effect of Information Percolation in CAV networks with different parameters and for different scenarios, such as \cite{talebpour2018effect} and \cite{shang2017agent}.